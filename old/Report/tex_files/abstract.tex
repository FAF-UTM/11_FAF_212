\chapter*{Abstract}

"MeDSL" is a Domain-Specific Language developed by Team 11, which consists of Brînza Cristian, Lupașcu Felicia, Popescu Nichifor, and Zaica Maia, students from the Technical University of Moldova.
Our report includes the following chapters: Introduction, Domain Analysis, Language Overview, and Language Design and Implementation.
The aim of this project was to design and implement a Domain Specific Language for analyzing medical results in the healthcare industry. Medical data analysis is a critical component of healthcare delivery. Using Domain Specific Language can significantly improve the efficiency and accuracy of this process,  faster processing times, and reduced error rates. This report provides an overview of the development of Domain Specific Language and its potential benefits in solving problems associated with medical data analysis.
It can be customized to meet the specific needs of a particular medical specialty or area of research, and it can automate many of the tasks involved in data analysis, such as data extraction, cleaning, and transformation. This can save time and reduce the risk of errors. It can also increase the transparency and trustworthiness of the analysis, making it easier for other researchers to replicate the results. Overall, a domain-specific language is a flexible, efficient, and accurate way to analyze medical data and generate reports that are tailored to the specific needs of doctors, researchers, and policymakers.

\vspace{0.5cm}

\emph{\textbf{Keywords: } analysis, data, domain-specific language, medical results, specific problem.}

